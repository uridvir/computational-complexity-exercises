\documentclass[12pt]{article}

\title{Chapter \#0}
\author{Uri Dvir \\ Computational Complexity - A Modern Approach (2016 ed.)}
\date{completed December 22, 2023}

\usepackage{amssymb}
\usepackage{amsmath}

\newcommand{\qed}{\null \hfill $\square$}

\renewcommand{\labelenumi}{(\alph{enumi})}

\begin{document}

\maketitle

\section*{Exercise 1}
% For each of the following pairs of functions f , g determine whether f = o(g), g = o(f)
%or f = O(g). If f = o(g) then find the first number n such that f(n) < g(n):

\begin{enumerate}
\item{
% f(n) = n^2, g(n) = 2n^2 + 100 sqrt(n)
$f = O(g)$ since $f(n) < g(n)$ for all $n$, and thus for any sufficiently large $n$ (any $n$ at all) we have $n^2 < 2n^2 + 100\sqrt{n} \Rightarrow f(n) \leq c \cdot g(n)$ with $c = 1$.

\noindent
While it may be obvious that neither $f = o(g)$ nor $g = o(f)$ are true in this case, I'll prove it for completeness.
\begin{itemize}
\item{Suppose $f = o(g)$. Consider $\epsilon = 0.01$. Thus
$$
n^2 \leq 0.02n^2 + \sqrt{n} \Rightarrow 0.98 \leq \sqrt{n}/n^2
$$
This fails for even small $n$ (such as $n=2$). So $f \neq o(g)$.}
\item{Clearly $g \neq o(f)$ since $f$ is strictly less than $g$ so even $\epsilon = 1$ fails.}
\end{itemize}
}
\item{
% f(n) = n^100, g(n) = 2^(n/100)
We can write $g(n) = (2^{0.01})^n$. We will show that $f = o(g)$. Let $\epsilon > 0$ and let $c = 2^{0.01}$. We will show that sufficiently large $n$ can satisfy
$$
n^{100} \leq \epsilon c^n
$$
Taking the logarithm gives
$$
100 \log n \leq \log \epsilon + n \log c
\Rightarrow
\frac{\log n - 0.01 \log \epsilon}{n} \leq 0.01\log c
$$
We know from analysis that $\log(n)/n$ will approach 0 as $n$ goes to infinity, so the left-hand side must also approach 0. Thus sufficiently large $n$ will indeed satisfy the condition such that $f = o(g)$. The first $n$ that satisfies $f(n) < g(n)$ is $n=0$.
}
\end{enumerate}

\end{document}