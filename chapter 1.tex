\documentclass[12pt]{article}

\title{Chapter 1}
\author{Uri Dvir \\ Computational Complexity - A Modern Approach (2016 ed.)}
\date{worked on 12/23/2023}

\usepackage{amssymb}
\usepackage{amsmath}

\newcommand{\qed}{\null \hfill $\square$}

\renewcommand{\labelenumi}{(\alph{enumi})}
\renewcommand{\labelenumii}{\arabic{enumii}.}

\newcommand{\Unit}[1]{\,\text{#1} \,}

\begin{document}

\maketitle

\section*{Exercise 1}

\begin{enumerate}
\item{
To compute the addition function, we will describe a Turing machine with 1 input tape, 3 work tapes, and one output tape. The input $\langle x, y \rangle$ is encoded with deliminator $\#$ by $0 \rightarrow 00$, $1 \rightarrow 11$, $\# \rightarrow 01$.
\begin{enumerate}
\item{Seek right on the input tape until we arrive at the first $\square$.}
\item{Copy $y$ backwards to work tape 1.}
\item{Copy $x$ backwards to work tape 2.}
\item{Seek left on work tapes 1 and 2 until we arrive at $\triangleright$ on both.}
\item{Until both work tapes 1 and 2 read $\square$ and there is no carry, go to the next step. Otherwise, skip the next step.}
\item{Add the bits from work tape 1 and work tape 2 together (treating $\square$ as 0), write the result to work tape 3, and set the carry appropriately. Seek right on all three tapes and repeat the previous step.}
\item{Copy work tape 3 backwards to output and halt.}
\end{enumerate}
}
\pagebreak
\item{
For the multiplication function, we will have 4 work tapes. These are $\tau_x$, $\tau_y$, $\tau_{count}$, and $\tau_{sum}$. The encoding is the same as before. We will use bit-shifting to do the multiplication.
\begin{enumerate}
\item{Seek right to $\square$ on the input tape.}
\item{Copy $y$ backwards to $\tau_y$.}
\item{Copy $x$ backwards to $\tau_x$.}
\item{Seek to $\triangleright$ on $\tau_x$ and $\tau_y$, then seek right on $\tau_x$.}
\item{Add the bits of $\tau_{count}$ (leading zeros) to $\tau_{sum}$ and then seek to $\triangleright$ for $\tau_{count}$ but not for $\tau_{sum}$.}
\item{If $\tau_x$ reads 1, add the bits of $\tau_y$ to $\tau_{sum}$, directly after the leading zeros added in the last step.}
\item{Seek right on $\tau_x$, and skip to step 10 if we read $\square$.}
\item{Seek right to $\square$, write 0, then seek to $\triangleright$ on $\tau_{count}$.}
\item{Seek to $\triangleright$ for $\tau_{sum}$ and repeat step 5.}
\item{Seek right to $\square$ on $\tau_{sum}$ then copy $\tau_{sum}$ backwards to output, and halt.}
\end{enumerate}
}
\end{enumerate}

\section*{Exercise 2}

%Complete the proof of Claim 1.5 by writing down explicitly the description of the
% machine ˜M.
To simulate $M$, the machine $\tilde{M}$ will (1) read $\log|\Gamma|$ symbols from the tapes (seeking right), (2) write $\log|\Gamma|$ symbols to the tapes (seeking left), and then (3) seek left or right by $\log|\Gamma|$ positions, or not at all, on each tape. After the seek phase, we go back to the read phase.
\begin{itemize}
\item{To read $\log|\Gamma|$ symbols, we have a state for every prefix bit-string. We can represent these by $S = \Cup_{i=0}^{\log|\Gamma| - 1} \tilde{\Gamma}^i$. Our states for all $k$ tapes are $S^k$.}
\item{To write $\log|\Gamma|$ symbols to $k$ tapes, we have a sequence of states for every symbol $\sigma \in \Gamma$ we just read, represented by $\Gamma^k \times \mathbb{N}_{\log|\Gamma|}$.}
\item{The seeking states have the same structure as those for writes, $\Gamma^k \times \mathbb{N}_{\log|\Gamma|}$.}
\end{itemize}
The behavior of each read, write, or seek state of $\tilde{M}$ is also determined by the active state $q$ of $M$. We can write these states as
$$
Q \times \left(
(\textsf{READ}, S^k) \cup
(\textsf{WRITE}, \Gamma^k \times \mathbb{N}_{\log|\Gamma|}) \cup
(\textsf{SEEK}, \Gamma^k \times \mathbb{N}_{\log|\Gamma|})
\right)
$$
We can now write a transition function for $\tilde{M}$. To represent a $\log|\Gamma|$-long sequence, we will use superscript $*$. Given an encoding function $\textsf{enc} \,:\, \Gamma \rightarrow \tilde{\Gamma}^{\log|\Gamma|}$,
\begin{multline*}
\tilde{\delta}
((q, \textsf{READ}, s^*), \tilde{\sigma}^*) = \\
\begin{cases}
(q, \textsf{READ}, s^* \tilde{\sigma}^*), \tilde{\sigma}^*, \textsf{R}^k &
\text{if } |s\tilde{\sigma}| < \log|\Gamma| \\
(q, \textsf{WRITE}, \textsf{enc}^{-1}(s^* \tilde{\sigma}^*), \log|\Gamma|), \tilde{\sigma}^*, \textsf{S}^k &
\text{if } |s\tilde{\sigma}| = \log|\Gamma|
\end{cases}
\end{multline*}
\begin{multline*}
\tilde{\delta}
((q, \textsf{WRITE}, \sigma^*, j), \tilde{\sigma}^*) = \\
\begin{cases}
(q, \textsf{WRITE}, \sigma^*, j-1), \textsf{enc}(\delta(q,\sigma^*)_2)_j, \textsf{L}^k &
\text{if } j > 1 \\
(q, \textsf{SEEK}, \sigma^*, \log|\Gamma|), \textsf{enc}(\delta(q,\sigma^*)_2)_1, \textsf{S}^k &
\text{if } j = 1
\end{cases}
\end{multline*}
$$
\tilde{\delta}
((q, \textsf{SEEK}, \sigma^*, j), \tilde{\sigma}^*) =
\begin{cases}
(q, \textsf{SEEK}, \sigma^*, j-1),  \tilde{\sigma}^*, \delta(q,\sigma^*)_3 &
\text{if } j > 1 \\
(\delta(q,\sigma^*)_1, \textsf{READ}, \epsilon^*), \tilde{\sigma}^*, \textsf{S}^k &
\text{if } j = 1
\end{cases}
$$
To handle the $\square$ symbol correctly, we simply define $\textsf{enc}(\square) = \square^{\log|\Gamma|}$. I'm not sure yet how to deal with $\triangleright$: one idea is to place $\triangleright^{\log|\Gamma|}$ at the start of each tape and then use the same approach we did with $\square$. We could also implement a special case. 

\noindent
Either way, $\tilde{M}$ starts off with some setup then enters $(q_{start}, \textsf{READ}, \epsilon^*)$. We then assign $(q_{halt}, \textsf{READ}, \epsilon^*)$ as our halt state. Therefore a computation takes $3\log|\Gamma|T(n)$ plus some setup steps, which is within $4\log|\Gamma|T(n)$ steps.

\section*{Exercise 3}

%Complete the proof of Claim 1.6 (by explicitly defining \tilde{M}).

\begin{multline*}
\tilde{\delta}((q, \textsf{READ}, s, i), \tilde{\sigma}) = \\
\begin{cases}
(q, \textsf{WRITE}, s, \textsf{dehat}(s), \textsf{dec}(i), \tilde{\sigma}, \textsf{L} & \text{if } \textsf{count}(s, \square) = 0 \\
(q, \textsf{READ}, \textsf{insert}(s,\tilde{\sigma},i), \textsf{inc}(i)), \tilde{\sigma}, \textsf{R} & \text{else if } \textsf{hat}(\tilde{\sigma}) = \tilde{\sigma} \\
(q, \textsf{READ}, s, \textsf{inc}(i)), \tilde{\sigma}, \textsf{R} & \text{else if } \textsf{hat}(\tilde{\sigma}) \neq \tilde{\sigma}
\end{cases}
\end{multline*}
\begin{multline*}
\tilde{\delta}((q, \textsf{WRITE}, s, s', i), \tilde{\sigma}) = \\
\begin{cases}
(q, \textsf{READ}, \square^k, \textsf{inc}(i)), \tilde{\sigma}, \textsf{R} & \text{if } s = \square^k \\
(q, \textsf{SEEK1}, [\delta(q, s')_3]_i, 0, i), [\delta(q, s')_2]_i, [\delta(q, s')_3]_i & \text{else if } \textsf{hat}(\tilde{\sigma}) = \tilde{\sigma} \\
(q, \textsf{WRITE}, s, s', \textsf{dec}(i)), \tilde{\sigma}, \textsf{L} & \text{else if } \textsf{hat}(\tilde{\sigma}) \neq \tilde{\sigma}
\end{cases}
\end{multline*}

\end{document}