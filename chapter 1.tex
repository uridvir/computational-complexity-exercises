\documentclass[12pt]{article}

\title{Chapter 1}
\author{Uri Dvir \\ Computational Complexity - A Modern Approach (2016 ed.)}
\date{worked on 12/23/2023}

\usepackage{amssymb}
\usepackage{amsmath}

\newcommand{\qed}{\null \hfill $\square$}

\renewcommand{\labelenumi}{(\alph{enumi})}
\renewcommand{\labelenumii}{\arabic{enumii}.}

\newcommand{\Unit}[1]{\,\text{#1} \,}

\begin{document}

\maketitle

\section*{Exercise 1}

\begin{enumerate}
\item{
To compute the addition function, we will describe a Turing machine with 1 input tape, 3 work tapes, and one output tape. The input $\langle x, y \rangle$ is encoded with deliminator $\#$ by $0 \rightarrow 00$, $1 \rightarrow 11$, $\# \rightarrow 01$.
\begin{enumerate}
\item{Seek right on the input tape until we arrive at the first $\square$.}
\item{Copy $y$ backwards to work tape 1.}
\item{Copy $x$ backwards to work tape 2.}
\item{Seek left on work tapes 1 and 2 until we arrive at $\triangleright$ on both.}
\item{Until both work tapes 1 and 2 read $\square$ and there is no carry, go to the next step. Otherwise, skip the next step.}
\item{Add the bits from work tape 1 and work tape 2 together (treating $\square$ as 0), write the result to work tape 3, and set the carry appropriately. Seek right on all three tapes and repeat the previous step.}
\item{Copy work tape 3 backwards to output and halt.}
\end{enumerate}
}
\end{enumerate}

\end{document}