\documentclass[12pt]{article}

\title{Chapter 1}
\author{Uri Dvir \\ Computational Complexity - A Modern Approach (2016 ed.)}
\date{worked on 12/23/2023}

\usepackage{amssymb}
\usepackage{amsmath}

\newcommand{\qed}{\null \hfill $\square$}

\renewcommand{\labelenumi}{(\alph{enumi})}
\renewcommand{\labelenumii}{\arabic{enumii}.}

\newcommand{\Unit}[1]{\,\text{#1} \,}

\begin{document}

\maketitle

\section*{Exercise 1}

\begin{enumerate}
\item{
To compute the addition function, we will describe a Turing machine with 1 input tape, 3 work tapes, and one output tape. The input $\langle x, y \rangle$ is encoded with deliminator $\#$ by $0 \rightarrow 00$, $1 \rightarrow 11$, $\# \rightarrow 01$.
\begin{enumerate}
\item{Seek right on the input tape until we arrive at the first $\square$.}
\item{Copy $y$ backwards to work tape 1.}
\item{Copy $x$ backwards to work tape 2.}
\item{Seek left on work tapes 1 and 2 until we arrive at $\triangleright$ on both.}
\item{Until both work tapes 1 and 2 read $\square$ and there is no carry, go to the next step. Otherwise, skip the next step.}
\item{Add the bits from work tape 1 and work tape 2 together (treating $\square$ as 0), write the result to work tape 3, and set the carry appropriately. Seek right on all three tapes and repeat the previous step.}
\item{Copy work tape 3 backwards to output and halt.}
\end{enumerate}
}
\pagebreak
\item{
For the multiplication function, we will have 4 work tapes. These are $\tau_x$, $\tau_y$, $\tau_{count}$, and $\tau_{sum}$. The encoding is the same as before. We will use bit-shifting to do the multiplication.
\begin{enumerate}
\item{Seek right to $\square$ on the input tape.}
\item{Copy $y$ backwards to $\tau_y$.}
\item{Copy $x$ backwards to $\tau_x$.}
\item{Seek to $\triangleright$ on $\tau_x$ and $\tau_y$, then seek right on $\tau_x$.}
\item{Add the bits of $\tau_{count}$ (leading zeros) to $\tau_{sum}$ and then seek to $\triangleright$ for $\tau_{count}$ but not for $\tau_{sum}$.}
\item{If $\tau_x$ reads 1, add the bits of $\tau_y$ to $\tau_{sum}$, directly after the leading zeros added in the last step.}
\item{Seek right on $\tau_x$, and skip to step 10 if we read $\square$.}
\item{Seek right to $\square$, write 0, then seek to $\triangleright$ on $\tau_{count}$.}
\item{Seek to $\triangleright$ for $\tau_{sum}$ and repeat step 5.}
\item{Seek right to $\square$ on $\tau_{sum}$ then copy $\tau_{sum}$ backwards to output, and halt.}
\end{enumerate}
}
\end{enumerate}

\section*{Exercise 2}

%Complete the proof of Claim 1.5 by writing down explicitly the description of the
% machine ˜M.
To simulate $M$, the machine $\tilde{M}$ will (1) read $\log|\Gamma|$ symbols from the tapes (seeking right), (2) write $\log|\Gamma|$ symbols to the tapes (seeking left), and then (3) seek left or right by $\log|\Gamma|$ positions, or not at all, on each tape. After the seek phase, we go back to the read phase.
\begin{itemize}
\item{To read $\log|\Gamma|$ symbols, we have a state for every prefix bit-string. We can represent these by $S = \Cup_{i=0}^{\log|\Gamma| - 1} \tilde{\Gamma}^i$. Our states for all $k$ tapes are $S^k$.}
\item{To write $\log|\Gamma|$ symbols to $k$ tapes, we have a sequence of states for every symbol $\sigma \in \Gamma$ we just read, represented by $\Gamma^k \times \mathbb{N}_{\log|\Gamma|}$.}
\item{The seeking states have the same structure as those for writes, $\Gamma^k \times \mathbb{N}_{\log|\Gamma|}$.}
\end{itemize}
The behavior of each read, write, or seek state of $\tilde{M}$ is also determined by the active state $q$ of $M$. We can write these states as
$$
Q \times \left(
(\textsf{READ}, S^k) \cup
(\textsf{WRITE}, \Gamma^k \times \mathbb{N}_{\log|\Gamma|}) \cup
(\textsf{SEEK}, \Gamma^k \times \mathbb{N}_{\log|\Gamma|})
\right)
$$
We can now write a transition function for $\tilde{M}$. To represent a $\log|\Gamma|$-long sequence, we will use superscript $*$. Given an encoding function $\textsf{enc} \,:\, \Gamma \rightarrow \tilde{\Gamma}^{\log|\Gamma|}$,
\begin{multline*}
\tilde{\delta}
((q, \textsf{READ}, s^*), \tilde{\sigma}^*) = \\
\begin{cases}
(q, \textsf{READ}, s^* \tilde{\sigma}^*), \tilde{\sigma}^*, \textsf{R}^k &
\text{if } |s\tilde{\sigma}| < \log|\Gamma| \\
(q, \textsf{WRITE}, \textsf{enc}^{-1}(s^* \tilde{\sigma}^*), \log|\Gamma|), \tilde{\sigma}^*, \textsf{S}^k &
\text{if } |s\tilde{\sigma}| = \log|\Gamma|
\end{cases}
\end{multline*}
\begin{multline*}
\tilde{\delta}
((q, \textsf{WRITE}, \sigma^*, j), \tilde{\sigma}^*) = \\
\begin{cases}
(q, \textsf{WRITE}, \sigma^*, j-1), \textsf{enc}(\delta(q,\sigma^*)_2)_j, \textsf{L}^k &
\text{if } j > 1 \\
(q, \textsf{SEEK}, \sigma^*, \log|\Gamma|), \textsf{enc}(\delta(q,\sigma^*)_2)_1, \textsf{S}^k &
\text{if } j = 1
\end{cases}
\end{multline*}
$$
\tilde{\delta}
((q, \textsf{SEEK}, \sigma^*, j), \tilde{\sigma}^*) =
\begin{cases}
(q, \textsf{SEEK}, \sigma^*, j-1),  \tilde{\sigma}^*, \delta(q,\sigma^*)_3 &
\text{if } j > 1 \\
(\delta(q,\sigma^*)_1, \textsf{READ}, \epsilon^*), \tilde{\sigma}^*, \textsf{S}^k &
\text{if } j = 1
\end{cases}
$$
To handle the $\square$ symbol correctly, we simply define $\textsf{enc}(\square) = \square^{\log|\Gamma|}$. For $\triangleright$, we define some special cases:
$$
\forall i \ni [\tilde{\sigma}^*]_i = \triangleright 
\,;\,
[\tilde{\delta}((q, \textsf{READ}, s^*), \tilde{\sigma}^*)_3]_i := \textsf{S}
$$
$$
\forall i, j \ni [\tilde{\sigma}^*]_i = \triangleright, j \neq 1 
\,;\,
[\tilde{\delta}((q, \textsf{SEEK}, \sigma^*, j), \tilde{\sigma}^*)_3]_i := \textsf{S}
$$
$$
\forall i \ni [\tilde{\sigma}^*]_i = \triangleright  
\,;\,
[\tilde{\delta}((q, \textsf{SEEK}, \sigma^*, 1), \tilde{\sigma}^*)_3]_i := [\delta(q, \sigma^*)_3]_i
$$
The machine $\tilde{M}$ has the start state $(q_{start}, \textsf{READ}, \epsilon^*)$ and the halt state $(q_{halt}, \textsf{READ}, \epsilon^*)$. Therefore a computation takes $3\log|\Gamma|T(n) < 4\log|\Gamma|T(n)$ steps.

\section*{Exercise 3}

%Complete the proof of Claim 1.6 (by explicitly defining \tilde{M}).

First, $\tilde{M}$ will copy the input into the encoded section of the tape. To simulate each step of $M$, we sweep right and left 4 times.
\begin{itemize}
\item{Read the ``\^\, type" symbols into the registers (encode them in $Q$). Compute $\delta$ and also store this in the registers.}
\item{Write the new symbols to the tape, in hat form to preserve position.}
\item{Perform head adjustments for right seeks only.}
\item{Perform head adjustments for left seeks only.}
\end{itemize}
To make this ``oblivious," we expand the encoded section by exactly $k$ positions every cycle. This makes the head position of $\tilde{M}$ completely predictable. The tape will have the configuration $\triangleright \texttt{ input } \triangle \triangleright^k ...\, \triangle$. We use $\triangle$ to mark the ends of the encoded section for simplicity. 

\noindent
We also define some functions. $\textsf{hat}$ and $\textsf{dehat}$ add and remove the ``hat" (\^\,) from symbols, $\textsf{insert}(s, \sigma, i)$ returns $s$ with $\sigma$ inserted at position $i$, and $\textsf{inc}$ and $\textsf{dec}$ increment and decrement $1 \leq i \leq k$ such that $\textsf{inc}(k) = 1$ and $\textsf{dec}(1) = k$.
\begin{multline*}
\tilde{\delta}((q, \textsf{COPY1}, \phi, i), \tilde{\sigma}) = \\
\begin{cases}
(q, \textsf{COPY1}, 1, 0), \tilde{\sigma}, \textsf{R} & \text{if } \phi = 1, i = 0, \tilde{\sigma} \neq \square \\
(q, \textsf{COPY1}, 2, 1), \triangle, \textsf{R} & \text{if } \phi = 1, i = 0, \tilde{\sigma} = \square \\
(q, \textsf{COPY1}, 2, i+1), \textsf{hat}(\triangleright), \textsf{R} & \text{if } \phi = 2, i \neq k, \tilde{\sigma} = \square \\
(q, \textsf{COPY1}, 3, 1), \textsf{hat}(\triangleright), \textsf{R} & \text{if } \phi = 2, i = k, \tilde{\sigma} = \square \\
(q, \textsf{COPY1}, 3, i+1), \square, \textsf{R} & \text{if } \phi = 3, i \neq k, \tilde{\sigma} = \square \\
(q, \textsf{COPY1}, 4, 0), \triangle, \textsf{R} & \text{if } \phi = 3, i = k, \tilde{\sigma} = \square \\
(q, \textsf{COPY1}, 4, 0), \tilde{\sigma}, \textsf{L} & \text{if } \phi = 4, i = 0, \tilde{\sigma} \neq \triangleright \\
(q, \textsf{COPY2}, ...), \triangleright, \textsf{R} & \text{if } \phi = 4, i = 0, \tilde{\sigma} = \triangleright
\end{cases}
\end{multline*}
\begin{multline*}
\tilde{\delta}((q, \textsf{COPY2}, \phi, i, \sigma), \tilde{\sigma}) = \\
\begin{cases}
(q, \textsf{COPY2}, 1, 0, \tilde{\sigma}), \textsf{hat}(\tilde{\sigma}), \textsf{R} & \text{if } \phi = 1, i = 0, \sigma = \square, \tilde{\sigma} \neq \triangle \\
(q, \textsf{COPY2}, 1, 0, \sigma), \tilde{\sigma}, \textsf{R} & \text{if } \phi = 1, i = 0, \sigma \neq \square, \tilde{\sigma} \neq \triangle \\
(q, \textsf{COPY2}, 2, 1, \sigma), \triangle, \textsf{R} & \text{if } \phi = 1, i = 0, \sigma \neq \square, \tilde{\sigma} = \triangle \\
(q, \textsf{COPY2}, 2, 2, \sigma), \tilde{\sigma}, \textsf{R} & \text{if } \phi = 2, i = 1, \sigma \neq \square, \tilde{\sigma} \neq \square \\
(q, \textsf{COPY2}, 2, \textsf{inc}(i), \sigma), \tilde{\sigma}, \textsf{R} & \text{if } \phi = 2, i \neq 1, \sigma \neq \square, \tilde{\sigma} \in \{\square, \triangle\} \\
(q, \textsf{COPY2}, 3, 0, \square), \sigma, \textsf{L} & \text{if } \phi = 2, i = 1, \sigma \neq \square, \tilde{\sigma} = \square \\
(q, \textsf{COPY2}, 1, 0, \square), \textsf{dehat}(\tilde{\sigma}), \textsf{R} & \text{if } \phi = 3, i = 0, \sigma = \square, \textsf{hat}(\tilde{\sigma}) = \tilde{\sigma} \\
(q, \textsf{READ1}, \square^k, 1), \triangle, \textsf{R} & \text{if } \phi = 1, i = 0, \sigma = \square, \tilde{\sigma} = \triangle
\end{cases}
\end{multline*}
\begin{multline*}
\tilde{\delta}((q, \textsf{READ1}, s, i), \tilde{\sigma}) = \\
\begin{cases}
(q, \textsf{READ2}, s, 1), \square, \textsf{R} & \text{if } \tilde{\sigma} = \triangle, i = k \\
(q, \textsf{READ1}, \textsf{insert}(s,\textsf{dehat}(\tilde{\sigma}),i), \textsf{inc}(i)), \tilde{\sigma}, \textsf{R} & \text{else if } \textsf{hat}(\tilde{\sigma}) = \tilde{\sigma} \\
(q, \textsf{READ1}, s, \textsf{inc}(i)), \tilde{\sigma}, \textsf{R} & \text{else if } \textsf{hat}(\tilde{\sigma}) \neq \tilde{\sigma}
\end{cases}
\end{multline*}
$$
\tilde{\delta}((q, \textsf{READ2}, s, i), \tilde{\sigma}) =
\begin{cases}
(q, \textsf{WRITE}, \delta(q, s), k-1), \triangle, \textsf{L} & \text{if } i = k \\
(q, \textsf{READ2}, s, \textsf{inc}(i)), \square, \textsf{R} & \text{if } i \neq k \\
\end{cases}
$$
\begin{multline*}
\tilde{\delta}((q, \textsf{WRITE}, q', s', d, i), \tilde{\sigma}) = \\
\begin{cases}
(q, \textsf{RSEEK}, q', s', d, 1), \triangle, \textsf{R} & \text{if } \tilde{\sigma} = \triangle, i = k \\
(q, \textsf{WRITE}, q', s', d, \textsf{dec}(i)), \textsf{hat}([s']_i), \textsf{L} & \text{else if } \textsf{hat}(\tilde{\sigma}) = \tilde{\sigma} \\
(q, \textsf{WRITE}, q', s', d, \textsf{dec}(i)), \tilde{\sigma}, \textsf{L} & \text{else if } \textsf{hat}(\tilde{\sigma}) \neq \tilde{\sigma}
\end{cases}
\end{multline*}
\begin{multline*}
\tilde{\delta}((q, \textsf{RSEEK}, q', s', d, b, i), \tilde{\sigma}) = \\
\begin{cases}
(q, \textsf{LSEEK}, \delta', b, k-1), \triangle, \textsf{L} & \text{if } \tilde{\sigma} = \triangle, i = k \\
(q, \textsf{RSEEK}, \delta', \textsf{insert}(b, 1, i), \textsf{inc}(i)), \textsf{dehat}(\tilde{\sigma}), \textsf{R} & \text{if } \textsf{hat}(\tilde{\sigma}) = \tilde{\sigma}, d_i = \textsf{R} \\
(q, \textsf{RSEEK}, \delta', \textsf{insert}(b, 0, i), \textsf{inc}(i)), \textsf{hat}(\tilde{\sigma}), \textsf{R} & \text{if } d_i = \textsf{R}, b_i = 1
\end{cases}
\end{multline*}
\begin{multline*}
\tilde{\delta}((q, \textsf{LSEEK}, q', s', d, b, i), \tilde{\sigma}) = \\
\begin{cases}
(q', \textsf{READ1}, \square^k, 1), \triangle, \textsf{R} & \text{if } \tilde{\sigma} = \triangle, i = k \\
(q, \textsf{LSEEK}, \delta', \textsf{insert}(b, 1, i), \textsf{dec}(i)), \textsf{dehat}(\tilde{\sigma}), \textsf{L} & \text{if } \textsf{hat}(\tilde{\sigma}) = \tilde{\sigma}, d_i = \textsf{L} \\
(q, \textsf{LSEEK}, \delta', \textsf{insert}(b, 0, i), \textsf{dec}(i)), \textsf{hat}(\tilde{\sigma}), \textsf{L} & \text{if } d_i = \textsf{L}, b_i = 1
\end{cases}
\end{multline*}
Just like the last exercise, the start symbol causes us a bit of grief. Thankfully, we only need one special case, for left seek. To prevent seeking past our emulated start symbol,
$$
\tilde{\delta}((q, \textsf{LSEEK}, q', s', d, b, i), \textsf{hat}(\triangleright)) := (q, \textsf{LSEEK}, q', s', d, b, \textsf{dec}(i)), \textsf{hat}(\triangleright), \textsf{L}
$$
We have start state $(q_{start}, \textsf{COPY1}, 1, 0)$ and halt state $(q_{halt}, \textsf{READ1}, \square^k, 1)$. The copy takes $O(n^2)$ steps, and the encoded section is $k(i+1)$ wide for the $i$-th step. Therefore each sweep can take no more than $k(T(n) + 1)$ steps. Sweeping 4 times for $T(n)$ cycles gives $4k(T^2(n) + T(n)) < 5k \cdot (T(n))^2$ steps.

\section*{Exercise 5}

% Prove that for time-constructible T that any language L in DTIME(T) can be decided in time O(T^2) by an oblivious 2-tape TM.

Since $L \in \textbf{DTIME}(T(n))$, there exists a Turing machine $M$ which decides $L$ in $O(T(n))$ time. From Claim 1.6 (which we proved in Exercise 3), we know that $M$ can be simulated by a single-tape machine $\tilde{M}$ in $O(T^2(n))$ time. Our proof for Exercise 3 constructs an oblivious TM. Furthermore, a two-tape version of $\tilde{M}$ with equal running time is trivial to construct: we can write the input to the work tape during the copy phase with no extra steps. So a two-tape oblivious machine exists which can decide $L$ in $O(T^2(n))$ time.

\section*{Exercise 6}

% Improve the bound of exercise 5 to O(T(n) log T(n))

We know there is a machine $M$ which decides $L$ in $O(T(n))$ time but we are not guaranteed that $M$ is oblivious.

\end{document}