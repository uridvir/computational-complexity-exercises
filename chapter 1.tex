\documentclass[12pt]{article}

\title{Chapter 1}
\author{Uri Dvir \\ Computational Complexity - A Modern Approach (2016 ed.)}
\date{worked on 12/23/2023}

\usepackage{amssymb}
\usepackage{amsmath}

\newcommand{\qed}{\null \hfill $\square$}

\renewcommand{\labelenumi}{(\alph{enumi})}
\renewcommand{\labelenumii}{\arabic{enumii}.}

\newcommand{\Unit}[1]{\,\text{#1} \,}

\begin{document}

\maketitle

\section*{Exercise 1}

\begin{enumerate}
\item{
To compute the addition function, we will describe a Turing machine with 1 input tape, 3 work tapes, and one output tape. The input $\langle x, y \rangle$ is encoded with deliminator $\#$ by $0 \rightarrow 00$, $1 \rightarrow 11$, $\# \rightarrow 01$.
\begin{enumerate}
\item{Seek right on the input tape until we arrive at the first $\square$.}
\item{Copy $y$ backwards to work tape 1.}
\item{Copy $x$ backwards to work tape 2.}
\item{Seek left on work tapes 1 and 2 until we arrive at $\triangleright$ on both.}
\item{Until both work tapes 1 and 2 read $\square$ and there is no carry, go to the next step. Otherwise, skip the next step.}
\item{Add the bits from work tape 1 and work tape 2 together (treating $\square$ as 0), write the result to work tape 3, and set the carry appropriately. Seek right on all three tapes and repeat the previous step.}
\item{Copy work tape 3 backwards to output and halt.}
\end{enumerate}
}
\pagebreak
\item{
For the multiplication function, we will have 4 work tapes. These are $\tau_x$, $\tau_y$, $\tau_{count}$, and $\tau_{sum}$. The encoding is the same as before. We will use bit-shifting to do the multiplication.
\begin{enumerate}
\item{Seek right to $\square$ on the input tape.}
\item{Copy $y$ backwards to $\tau_y$.}
\item{Copy $x$ backwards to $\tau_x$.}
\item{Seek to $\triangleright$ on $\tau_x$ and $\tau_y$, then seek right on $\tau_x$.}
\item{Add the bits of $\tau_{count}$ (leading zeros) to $\tau_{sum}$ and then seek to $\triangleright$ for $\tau_{count}$ but not for $\tau_{sum}$.}
\item{If $\tau_x$ reads 1, add the bits of $\tau_y$ to $\tau_{sum}$, directly after the leading zeros added in the last step.}
\item{Seek right on $\tau_x$, and skip to step 9 if we read $\square$.}
\item{Seek right to $\square$, write 0, then seek to $\triangleright$ on $\tau_{count}$.}
\item{Seek to $\triangleright$ for $\tau_{sum}$ and repeat step 5.}
\item{Seek right to $\square$ on $\tau_{sum}$ then copy $\tau_{sum}$ backwards to output, and halt.}
\end{enumerate}
}
\end{enumerate}

\end{document}