\documentclass[12pt]{article}

\title{Notes}
\author{Uri Dvir \\ Computational Complexity - A Modern Approach (2016 ed.)}

\usepackage{amssymb}
\usepackage{amsmath}

\newcommand{\qed}{\null \hfill $\square$}

\renewcommand{\labelenumi}{(\arabic{enumi}).}

\newcommand{\Unit}[1]{\,\text{#1} \,}

\begin{document}

\maketitle

\section*{Chapter 1}

\begin{enumerate}
\item{
A Turing machine (TM) with $k \geq 2$ tapes is defined as a tuple $(\Gamma, Q, \delta)$ where
\begin{enumerate}
\item{$\Gamma$ is a finite set of symbols, the alphabet of the machine. $\Gamma$ contains two special symbols, a start symbol $\triangleright$ and a blank symbol $\square$, as well as the normal ones like 0 and 1.}
\item{$Q$ is the set of states. There are two special states, the start state $q_{start}$ and the halt state $q_{halt}$.}
\item{$\delta \,:\, Q \times \Gamma^k \rightarrow Q \times \Gamma^{k-1} \times \{ L, S, R \}^k$ is the transition function. The TM reads the value under each tape head and then, depending on the state, writes to the writeable tapes, and then moves each tape head left, right, or not at all. The machine ends up in a new state. The transition function $\delta$ describes precisely how the TM does this for every configuration.}
\end{enumerate}
The TM's first tape is its input, which is read-only. Having three tapes, one for input, one for work, and one for output, is pretty common.
}
\item{
How does a Turing machine deal with the start symbol $\triangleright$? How exactly does it handle weird edge cases? Theses kind of minutiae are arbitrary since different variations of TMs are basically equivalent. However, it greatly simplifies my proofs if I have a consistent and rigid definition. Therefore I decree:
\begin{itemize}
\item{The start symbol $\triangleright$ at the leftmost position is read-only. If a TM tries to write something other than $\triangleright$, this action is legal but does nothing.}
\item{Seeking left out-of-bounds also does nothing.}
\item{A TM is free to write and erase $\triangleright$ at other positions, \textit{at its own peril}. In principle you could design a Turing machine that forgets where the ``real" start symbol is. That is objectively a skill issue, however.}
\end{itemize} 
}
\item{A TM computes a function $f$ if, given $x$ as input, it always has $f(x)$ on its output tape when it halts. It computes in $T$-time if the computation on $x$ requires at most $T(|x|)$ steps.}
\item{
A function $T \,:\, \mathbb{N} \rightarrow \mathbb{N}$ is time-constructible if $T(n) \geq n$ and there exists a Turing machine $M$ which computes $x \rightarrow T(|x|)$ in $T$-time. Basically, is there a Turing machine which computes its own $T$-time, in $T$-time? TODO: Update this definition when I get better intuition for it.
}
\item{
A Turing machine is \textit{oblivious} if its head movements only depend on the length of the input. So, for an input $x$, the location of each head on the $i$-th step is a function of $|x|$ and $i$. Every Turing machine can be simulated by an oblivious Turing machine.
}
\item{A language $L$ is in $\textbf{DTIME}(T(n))$ iff there exists a TM which takes $O(T(n))$ time to decide $L$.}
\end{enumerate}

\end{document}